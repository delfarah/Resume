\documentclass[margin, 10pt]{res}

\usepackage{helvet}
\usepackage[hidelinks]{hyperref}
\usepackage{changepage}   % for the adjustwidth environment
\usepackage{graphicx}
\usepackage{fancyhdr}
\renewcommand{\headrulewidth}{0pt}
 
\pagestyle{fancy}
\chead{
\begin{adjustwidth}{-3.5cm}{}
\moveleft.5\hoffset\centerline{\large\bf Masood Delfarah, Ph.D. Candidate}
\begin{tabular}{l r}
\\
\href{http://web.cse.ohio-state.edu/~delfarah.1/}{http://web.cse.ohio-state.edu/$\sim$delfarah.1}    & \hspace{2.5in}  \href{mailto:delfarah.1@osu.edu}{delfarah.1@osu.edu} \\
\href{https://www.linkedin.com/in/mdelfarah/}{https://www.linkedin.com/in/mdelfarah/}    & \hspace{2.5in}  {Tel: (614) 477-7344} \\
\end{tabular}
\\
\moveleft\hoffset\vbox{\hrule width\resumewidth height 0.5pt}\smallskip
\end{adjustwidth}
}
%\rfoot{\thepage}

\setlength{\headheight}{45.0pt}
\setlength{\footskip}{10.0pt}

\setlength{\textwidth}{5.1in} % Text width of the document
\setlength{\textheight}{8.1in} % Text width of the document

\begin{document}
\begin{resume}
\section{RESEARCH\\INTERESTS}

\begin{tabular}{l l}
\textbf{Monaural Speech Enhancement}    & \hspace{0.3in}   Automatic Speech Recognition\\ 
\textbf{Speech Dereverberation}    & \hspace{0.3in}    Deep Learning\\ 
Microphone Array Speech Processing  & \hspace{0.3in} Statistical Machine Learning
\end{tabular}

\section{EDUCATION}
\textbf{Department of Computer Science and Eng., The Ohio State University }  \\
\textbf{Ph.D. in Computer Eng.} \hfill Fall 2013 -- Present \\
\textbf{Supervisor:} Prof. DeLiang Wang

\textbf{School of Electrical and Computer Engineering, The University of Tehran}\\
\textbf{B.Sc. in Computer Eng.} \hfill Fall 2008 -- Spring 2013 \\
\textbf{GPA:} 16.57/20.00 (top 10\%).


\section{PUBLICATIONS AND PRESENTATIONS}
\textit{Journal papers:}
\begin{itemize}
\item Eric W. Healy, \textbf{Masood Delfarah}, Jordan L. Vasko, Brittney L. Carter, and DeLiang Wang, ``An algorithm to increase intelligibility for hearing-impaired listeners in the presence of a competing talker.'' \textit{The Journal of the Acoustical Society of America}, vol. 141, pp. 4230$-$4239, 2017.
\item \textbf{Masood Delfarah} and DeLiang Wang, ``Features for masking-based monaural speech separation in reverberant conditions.'' \textit{IEEE/ACM Transactions on Audio, Speech, and Language Processing}, vol. 25, pp. 1085$-$1094, 2017.
\item Maryam S. Mirian, \textbf{Masood Delfarah} , and Behzad Moshiri , ``Proposing a Unified Knowledge and Experience-based System using Information Fusion Approach to Facilitate the Disaster Management Process'' \textit{Disaster Management Knowledge Quarterly} (in Persian), vol. 2, pp. 215$-$227, 2012. 
\end{itemize}

\textit{Conference papers:}
\begin{itemize}
\item \textbf{Masood Delfarah} and DeLiang Wang, ``A feature study for masking-based reverberant speech separation.'' \textit{Proceedings of INTERSPEECH}, pp. 555$-$559, 2016.
\end{itemize}

\textit{Selected poster presentations:}
\begin{itemize}
\item Eric W. Healy, \textbf{Masood Delfarah}, Jordan L. Vasko, and Brittney L. Carter, and DeLiang Wang, ``Can a trained deep neural network be implemented into hearing technology?'' \textit{Acoustics '17 Boston}, 2017.
\item Eric W. Healy, \textbf{Masood Delfarah}, Jordan L. Vasko, Brittney L. Carter, and DeLiang Wang, ``An algorithm to increase intelligibility for hearing-impaired listeners in the presence of a competing talker'' \textit{Acoustics '17 Boston}, 2017.
\end{itemize}

\section{RESEARCH \\ EXPERIENCE}
\textit{Graduate Research:}
\begin{itemize}
\item Colaborated in a team to duplicate speaker independent speech separation methods such as deep clustering deep attractor network and permutation invariant training
\item Did augumated object detection and question answering
\item Used Matlab to do cochannel speaker identification in reverberant conditions
\item Implemented deep neural network to do speech separation in reverberant and anechoic conditions using matlab
	\subitem did feature study and SFFS and group lasso
	\subitem did anechoic training and set up experiments
	\subitem did reverberant two-stage Drev and denoise using tensorflow
\item did DNN-based cochannel speaker identification 
\end{itemize}

\textit{Undergraduate research:}
\begin{itemize}
\item \textbf{(BsC thesis)} Designing and implementing a learning style classifier for toddlers, based on cognitive traits of childrenand Information Processing Theories on the ELCK-12 dataset.
\item Proposing a knowlwdge-base for disaster management Superviser
\item Analyzed Persian Blogosphere to Obtain Social Network of Iranian Politicians and visualizing and graph clustering using \textit{Gephi}.
\end{itemize}

\section{PROFESSIONAL \\ EXPERIENCE}
\textit{Reviewer:}
\begin{itemize}
\item IEEE/ACM Transactions on Audio, Speech, and Language Processing
\item Speech Communication
\end{itemize}

\textit{Graduate Teaching Assistant, The Ohio State University:}
\begin{itemize}
\item Modeling and Problem Solving with Spreadsheets and Databases\hfill Spring 2017
\item Modeling and Problem Solving with Spreadsheets and Databases\hfill Spring 2014 
\item Foundations I: Discrete Structures\hfill Fall 2013
\end{itemize}
\textit{Undergraduate Teaching Assistant, The University of Tehran:}
\begin{itemize}
\item Design and Analysis of Algorithms\hfill Spring 2012 
\item Discrete Mathematics Course\hfill Spring 2012
\item Artificial intelligence\hfill Fall 2011
\end{itemize}


\section{COMPUTER \\ SKILLS}
\textbf{Proficient in C/C++}\\
Experienced in MATLAB, Python, and Java\\
Familiar with C\#, PHP, JavaScript, HTML, and SQL\\ 
Other skills: Bash, git, and \LaTeX

Machine Learning toolboxes: Tensorflow,
Caffe, htk, theano, MXNet


\end{resume}
\end{document}
