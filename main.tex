\documentclass[margin, 10pt]{res}

\usepackage{helvet}
\usepackage[hidelinks]{hyperref}
\usepackage{changepage}   % for the adjustwidth environment
\usepackage{graphicx}
\usepackage{fancyhdr}
\renewcommand{\headrulewidth}{0pt}
 
\pagestyle{fancy}
\chead{
\begin{adjustwidth}{-3.5cm}{}
\moveleft0.5\hoffset\centerline{\huge\bf Masood Delfarah}
\moveleft0.5\hoffset\centerline{\small \bf Ph.D. Candidate}
\begin{tabular}{l r}
\\
\href{http://web.cse.ohio-state.edu/~delfarah.1/}{http://web.cse.ohio-state.edu/$\sim$delfarah.1}    & \hspace{2.5in}  \href{mailto:delfarah.1@osu.edu}{delfarah.1@osu.edu} \\
\href{https://www.linkedin.com/in/mdelfarah/}{https://www.linkedin.com/in/mdelfarah}    & \hspace{2.5in}  {Tel: (614) 477-7344} \\
\end{tabular}
\\
\moveleft\hoffset\vbox{\hrule width\resumewidth height .5pt}\smallskip
\end{adjustwidth}
}
%\rfoot{\thepage}

%\setlength{\sectionwidth}{-30.0pt}

\setlength{\headheight}{45.0pt}
\setlength{\footskip}{20.0pt}

\setlength{\textwidth}{5.1in} % Text width of the document
\setlength{\textheight}{8.1in} % Text width of the document

\begin{document}
\begin{resume}

\section{EDUCATION}
\textbf{Department of Computer Science and Eng., The Ohio State University }  \\
\textbf{Ph.D. candidate in Computer Engineering} \hfill {\footnotesize Fall 2013 $-$ Present} \\
\textbf{M.Sc. in Computer Engineering} \hfill {\footnotesize Fall 2013 $-$ Spring 2018} \\
Perception and Neurodynamics Laboratory (PNL)\\
Supervisor: Professor DeLiang Wang

\textbf{School of Electrical and Computer Engineering, The University of Tehran}\\
\textbf{B.Sc. in Computer Engineering} \hfill {\footnotesize Fall 2008 $-$ Spring 2013} \\
GPA: 16.57/20.00 (top 10\%).


\section{RESEARCH\\INTERESTS}
\begin{tabular}{l l}
\textbf{Monaural Speech Enhancement}    & \hspace{0.3in}   Automatic Speech Recognition\\ 
\textbf{Speech Dereverberation}    & \hspace{0.3in}    Deep Learning\\ 
Microphone Array Speech Processing  & \hspace{0.3in} Statistical Machine Learning
\end{tabular}

\section{COMPUTER \\ SKILLS}
\textbf{C/C++}, \textbf{MATLAB}, \textbf{Python}, Java, and Unix shell script\\
Machine learning toolboxes: \textbf{Tensorflow},
Caffe, HTK, PyTorch, and MXNet\\
Other skills: Git and LaTeX


\section{RESEARCH \\ EXPERIENCE}
\textit{Graduate Research:}
\begin{itemize}
\item Feature study for two-talker speech separation in reverberant conditions:
	\begin{itemize}
	\item Utilized parallel computation and GPU servers on the Ohio Supercomputing Center for large-scale DNN training for speech separation.
	\item Investigated a wide range of acoustic-phonetic features and designed novel feature combinations based on feature selection methods.
	\end{itemize}

\item DNN-based two-talker separation algorithm:
	\begin{itemize}
	\item Designed and implemented two-talker separation algorithm.
	\item Deployed a development set to optimize the performance of the DNN by studying various architectures and regularization method.
	\item Collaborated with The Speech Psychoacoustics Laboratory at The Department of Speech and Hearing Science to perform speech intelligibility tests on human listeners.
	\item Perform statistical analysis on the test results and report substantial intelligibility improvement for hearing-impaired listeners.
	\end{itemize}

\item Designed and implemented a two-stage DNN to perform joint dereverberation and speech denoising.
\item Investigated two-talker speaker identification in reverberant mixtures
\item Studied microphone array methods for dereverberation of simulated and recorded reverberant speech signals.

\item Collaborated with lab members to study open-set speaker separation methods:
	\begin{itemize}
	\item Successfully implemented deep clustering, deep attractor network, and permutation invariant training algorithms
	\item Utilized distributed computation over a grid of nodes and GPU servers to perform data parallelism in Tensorflow.
	\item Evaluated performance of the algorithms in reverberant conditions.
	\end{itemize}

\item Performed pitch-tracking and speech segmentation based on the techniques in Computational Auditory Scene Analysis (CASA) using Java.
\item Evaluated effect of augmenting object detection into visual question answering (VQA) algorithms.
\item Studied transfer learning in reinforcement learning framework (ongoing project)

\end{itemize}

\textit{Undergraduate research:}
\begin{itemize}
\item \textbf{(B.Sc. Thesis)} Designed and implemented a decision tree to classify learning styles of toddlers, using the ECLS-K dataset provided by U.S. Department of Education. (Supervisor: Dr. Maryam S. Mirian)
\end{itemize}

\section{PROFESSIONAL \\ EXPERIENCE}
\textit{Reviewer:}
\begin{itemize}
\item IEEE/ACM Transactions on Audio, Speech, and Language Processing
\item Speech Communication
\end{itemize}

\textit{Graduate Teaching Assistant, The Ohio State University:}
\begin{itemize}
\item Modeling and Problem Solving with Spreadsheets and Databases\hfill {\footnotesize Spring 2017}
\item Modeling and Problem Solving with Spreadsheets and Databases\hfill {\footnotesize Spring 2014}
\item Foundations I: Discrete Structures\hfill {\footnotesize Fall 2013}
\end{itemize}
\textit{Undergraduate Teaching Assistant, The University of Tehran:}
\begin{itemize}
\item Design and Analysis of Algorithms\hfill {\footnotesize Spring 2012}
\item Discrete Mathematics\hfill {\footnotesize Spring 2012}
\item Artificial Intelligence\hfill {\footnotesize Fall 2011}
\end{itemize}



\section{PUBLICATIONS AND PRESENTATIONS}
\textit{Journal papers:}
\begin{itemize}
\item Eric W. Healy, \textbf{Masood Delfarah}, Jordan L. Vasko, Brittney L. Carter, and DeLiang Wang, ``An algorithm to increase intelligibility for hearing-impaired listeners in the presence of a competing talker'' \textit{The Journal of the Acoustical Society of America}, vol. 141, pp. 4230$-$4239, 2017.
\item \textbf{Masood Delfarah} and DeLiang Wang, ``Features for masking-based monaural speech separation in reverberant conditions'' \textit{IEEE/ACM Transactions on Audio, Speech, and Language Processing}, vol. 25, pp. 1085$-$1094, 2017.
\item Maryam S. Mirian, \textbf{Masood Delfarah}, and Behzad Moshiri, ``Proposing a Unified Knowledge and Experience-based System using Information Fusion Approach to Facilitate the Disaster Management Process'' \textit{Disaster Management Knowledge Quarterly} (in Persian), vol. 2, pp. 215$-$227, 2012. 
\end{itemize}

\textit{Conference papers:}
\begin{itemize}
\item \textbf{Masood Delfarah} and DeLiang Wang,``A feature study for masking-based reverberant speech separation'' \textit{Proceedings of INTERSPEECH-16}, pp. 555$-$559, 2016.
\end{itemize}

\textit{Selected poster presentations:}
\begin{itemize}
\item Eric W. Healy, \textbf{Masood Delfarah}, Jordan L. Vasko, and Brittney L. Carter, and DeLiang Wang, ``Can a trained deep neural network be implemented into hearing technology?'' \textit{Acoustics '17 Boston}, 2017.
\item Eric W. Healy, \textbf{Masood Delfarah}, Jordan L. Vasko, Brittney L. Carter, and DeLiang Wang, ``An algorithm to increase intelligibility for hearing-impaired listeners in the presence of a competing talker'' \textit{Acoustics '17 Boston}, 2017.
\end{itemize}


\end{resume}
\end{document}
